\newpage
\pdfbookmark[0]{Vortext}{frontmatter}
     \begin{flushright}
          \vspace*{-20mm}
          \includegraphics[width=\textwidth]{titel/CoverLogos.pdf}
     \end{flushright}
\vspace*{10mm} % mit \usepackage{printlen}: 1.8cm * \uselengthunit{cm}\printlength{\textwidth} / 16.01cm = -1.773cm -> passt nicht, 10mm durch optik
{\let\clearpage\relax}\thispagestyle{empty}\pdfbookmark[1]{Aufgabenstellung}{aufgabenstellung}

\begin{spacing}{1}        
\begin{center}
\large\textbf{Aufgabenstellung zur Projektarbeit/Studienarbeit/Bachelorarbeit/...}

\normalsize \Autor, Matrikelnummer \Matrikelnummer

\end{center}

\textbf{Allgemeines:}

(Beispieltext) An den Instituten für Antriebstechnik und Leistungselektronik und für Mechatronische Systeme wird in Kooperation ein neuartiges Aktuierungskonzept für die Endoskopie erarbeitet. Auf Basis von eigens entwickelten, elektromagnetischen, bistabilen Kippaktoren wird ein vollaktuierter Endoskopschaft aufgebaut. Durch die geeignete Anordnung einer Vielzahl von Aktoren kann eine quasi-kontinuierliche Bewegung des Endoskops erreicht werden. Durch die binäre Aktuierung der einzelnen Aktoren ist die insgesamt erreichbare Positionierbarkeit des Endeffektors des schlangenartigen Roboters sehr eingeschränkt und während des Vorschubs kommt es außerdem zu unerwünschten Bewegungen am Endeffektor.

\bigskip\textbf{Aufgabe:}

Um die unerwünschten Bewegungen auszugleichen und zudem eine höhere Positioniergenauigkeit des Endeffektors zu erreichen, soll eine -- beispielweise über Seilzüge -- kontinuierlich aktuierte Endeffektorplattform entworfen werden. Zusätzlich ist sie in den vorhandenen Versuchsaufbau zu integrieren und die Steuerung zu erweitern. Im Rahmen dieser Arbeit ergeben sich insbesondere die folgenden Aufgabenpunkte:

\begin{itemize}
	\item{Einarbeitungsphase (Projekt, vorhandener Prüfstand, vorhandene Modelle)}
	\item{Erarbeitung der Anforderungen an eine kontinuierlich aktuierte Endeffektorplattform}
	\item{Konstruktion der Endeffektorplattform}
	\item{Integration in den vorhanden Prüfstand}
	\item{Experimentelle Evaluation}
	\item{Dokumentation}
\end{itemize}

Die Bearbeitungszeit beträgt 300 Stunden.

\vfill
\begin{center}
\begin{tabular}{p{0.35\textwidth} p{0.35\textwidth}}
Ausgabe der Aufgabenstellung:&xx.xx.20xx\\
Abgabe der Arbeit spätestens am:&xx.xx.20xx\\
Erstprüfer: \\
Zweitprüfer: \\
Betreuer: 
\end{tabular}
\end{center}

\end{spacing}
% Leerseite einfügen
\cleardoublepage