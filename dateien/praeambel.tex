% ------------------------------------------------------------------------------------------
% Latex - Präambel
% hier werden alle wichtigen Dokumenteinstellungen getroffen, normalerweise müssen
% keine Änderungen mehr durchgeführt werden.
% Pakete die das Schriftbild, den Satzspiegel oder die Ränder verändern, dürfen
% nicht hinzugefügt werden, erst nach Abstimmung mit dem Betreuer.
% nützliche Pakete sind willkommen, bitte die Wiki entsprechend aktualisieren
% viele Pakete sind drin, die vielleicht nicht jeder braucht und das Kompilieren verlängern
% sollten diese das Layout nicht verändern, können diese natürlich auskommentiert werden.
% ------------------------------------------------------------------------------------------

% ------------------------------------------------------------------------------------------
% Dokumentenklasse definieren:
\ifdraft
    \documentclass[draft, 
                   paper=a4,
                   BCOR5mm,
                   fontsize=12pt, 
                   DIV13, 
                   headsepline, 
                   numbers=noenddot, 
                   %bibtotoc,
                   bibliography = totoc,
                   %version=first,
                   %smallheadings,
                   headings = small,
                   oneside]{scrbook}
\else
    \documentclass[paper=a4,
                   BCOR5mm,
                   fontsize=12pt, 
                   DIV13, 
                   headsepline, 
                   numbers=noenddot, 
                   %bibtotoc,
                   bibliography = totoc, 
                   %version=first,
                   %smallheadings,
                   headings = small,
                   headinclude=true,
                   footinclude=false,
                   %fleqn, % formeln links bündig
                   oneside]{scrbook}
\fi
% Standard-Koma-Dokument mit
%   Papierformat A4
%   Binderand 5 mm
%   Schrift 12-Punkt
%   Seitenlayout nach DIV, siehe scrguide.pdf text b/h rand oben/innen: 168,00 237,60 19,80 14,00 (ist raus)
%   Linie unter der Kopfzeile
%   Nummern ohne Punkt am Ende
%   Literaturverzeichnis im Inhaltsverzeichnis
%   Überschriften etwas kleiner als standard
%   weitere Informationen zum Koma-Skript: www.komascript.de

%---------------------------------------------------------------
% Basis-Pakete
%---------------------------------------------------------------
\usepackage{ifpdf}		%   Ueberprueft, ob LaTeX oder pdfLaTeX verwendet wird (nur ab MikTeX 2.5)
\usepackage[T1]{fontenc}        %   8-Bit-Fonts
\usepackage{textcomp,latexsym}  %   zusaetzliche Symbole
\usepackage[utf8]{inputenc}   %   Quelltext ist Latin-1 (d.h. Windows-) kodiert
\usepackage[ngerman]{babel}            %   neue deutsche Rechtschreibung
\usepackage{scrtime}            %   fuer die aktuelle Zeit
%\usepackage{scrdate}            %   fuer das aktuelle Datum
%\usepackage{natbib}             %   Literaturverweise mit (Autor Jahr) nach DIN
%\bibpunct[; ]{[}{]}{,}{a}{}{;}  %   eckige Klammern statt runde bei Zitaten
%\usepackage[T1]{url}            %   Web-Addressen auch mit T1-Encoding
%\urlstyle{tt}                   %   und in tt-Font
%\usepackage[activate=normal]{pdfcprot} % dient dem optischen Randausgleich bei kürzeren Zeichen wie '-', '.', ',', '!'

\usepackage{color}



\PassOptionsToPackage{x11names}{xcolor}
\usepackage{xcolor}


%---------------------------------------------------------------
% nützliche Zusatz-Pakete
%---------------------------------------------------------------
%\usepackage{makeidx}            %   Index (wenn gewuenscht)

\usepackage{setspace}           %   Zeilenabstand setzen. Befehl:
                                %   \begin{spacing{Wert}
                                %   Text
                                %   \end{spacing}
%\usepackage{placeins}           %   das Positionieren von float-Umgebungen bestimmen:
                                %   der Befehl \floatbarrier sorgt dafür, dass alle vorher eingegebenen
                                %   float-Umgebungen VOR diesem Befehl eingefügt werden.
                                
%\usepackage{subfig}          %   Bilder gruppieren, mehrere Bilder in einer Umgebung einfügen
                                %   Bsp.:
                                %   \begin{figure}
                                %   \centering
                                %    \subfloat[Text a]
                                %    {\includegraphics{bild a}}
                                %    \subfloat[text b]
                                %    {\includegraphics{bild b}}
                                %    \caption{Untertitel}
                                %    \label{fig.Beziechnung_subfigure} <- am besten mit fig. dann kann mit \bild{Bezeichnung_figure} referenziert werden
                                %   \end{figure}                                

%\usepackage{textcomp}           %   fuer Trademark und Copyright Zeichen
                                %   z.B \texttrademark, \textregistered, \texteuro etc.

\usepackage{tabularx}           %   erweiterte Funktionen für Tabellen

\usepackage{longtable}          %   Longtable-Package für Tabellen die länger als eine DIN-A4-Seite sind

\usepackage{colortbl}						%		farbige Tabellenzeilen

\usepackage{nicefrac}           %   Schräge Bruchstriche \nicefrac{Nenner}{Zähler}

%\usepackage{numprint}           %   Zahlen mit Einheiten näher an der Zahl/kein Umbruch und 1 000er Format \numprint[Einheit]{Zahl}

%\usepackage{mathcomp}           %   für \tcdegree (° Gradzeichen bei numprint) \numprint[\tcdegree]{180} für 180°

%\usepackage{hhline}             %   hhline-Package
                                %   Fuer komplexe Linienzuege in Tabellen \hhline{}
                                %   = Spalte mit doppelter Linie      | vertikale Linie die horizontale schneidet
                                %   - Spalte mit einfacher Linie      : vertikale Linie die von horizantaler unterbrochen ist
                                %   ~ Spalte ohne Linie               # doppelte horiz. geschnitten mit doppelter vert. Linie
                                %   t oberer Teil ->                  b unterer Teil einer doppelten Linie

%\usepackage[active]{srcltx}     %   bei verwendung v. \includeonly spring winedt jetzt bei fehlern an die richtige stelle

\usepackage{flafter}            %   Ein float-Objekt immer erst NACH seiner Definition platzieren

\usepackage{ifthen}             %   definiert \ifthenelse{}{}{}

%\usepackage{path}               %   dateinamen und pfade darstellen

%\usepackage{pdfsync}						%   Synchronisation mit SumatraPDF <<-- geht auch ohne

\usepackage{booktabs}

\usepackage{scrhack} % Löscht Warnungen von Packages wie hyperref, listings
\usepackage[ locale = DE,            % deutsch
            load-configurations=abbreviations, % zusätzliche Einheiten siehe manual
            per-mode=symbol,%-or-fraction,
            separate-uncertainty% 
            ]
            {siunitx}								% SI-Einheiten  
						\sisetup{detect-all} % aktuelle Schriftart nutzen (statt immer Mathemodus)

            
\usepackage{multirow}               % zwei Zeilen kombinieren (wie multicolumn)
\usepackage[babel]{microtype}       % schickeres Schriftbild -> aber längeres Kompilieren

\usepackage{blindtext}							% sinnlosen text einbauen
%\usepackage{subfigure}

\usepackage{newunicodechar}
%\newunicodechar{ }{~}


%\DeclareUnicodeCharacter{00A0}{~}



%---------------------------------------------------------------
% Grafik-Paket für eps bzw. pdf
%---------------------------------------------------------------
% überprüft ob Latex oder PDFLatex ausgeführt wird, dementsprechend werden pdf/jpg oder eps Bilder eingefügt
% sollen dvi oder pdf Dokumente erstellt werden müssen alle Bilder in beiden Formaten vorliegen
% hierfür gibt es Tools wie epstopdf oder jpgtoeps
 \ifdraft
    \usepackage[draft]{graphicx}
\else
    \ifpdf
        \usepackage{graphicx}
        \usepackage{pgfplots} % pgfplots zum plotten in LaTeX
    \else
        \usepackage[final]{graphicx}
    \fi
\fi
% von wo sollen die Grafiken kommen?

\graphicspath{{./bilder/}{./bilder/kinematik/}{./bilder/standdertechnik/}{./bilder/titel/}{./bilder/erlernen/}}

%\graphicspath{{./bilder/titel}}
%\graphicspath{{./bilder/einleitung}}
%\graphicspath{{./bilder/standdertechnik}}
%\graphicspath{{./bilder/kap3}}
%\graphicspath{{./bilder/kap4}}


%Einstellungen für pgfplots
\pgfplotsset{compat=1.3,
	%enlargelimits=auto,
	tick label style={font=\small,/pgf/number format/use comma}, %Komma nutzen in allen Axen
	%axis x line=center, %alle x-Achsen center
	%axis y line=center, %alle y-achsen center
	%every axis x label/.style={at={(current axis.right of origin)},anchor=west}, % achsenbeschriftung bei allen x-achsen rechts
	%every axis y label/.style={at={(current axis.above origin)},anchor=south},   % achsenbeschriftung bei allen y-achsen oben
	%every x tick scale label/.style={at={(current axis.right of origin)},anchor=north,yshift=-0.5em}, %positionierung des scale faktors geändert
	every axis legend/.append style={at={(0.5,1.03)},anchor=south,nodes=right,font=\small}, % Legende über der Grafik, schriftgröße small
	label style={font=\small} % label schriftgröße small
}
\newlength\tikzwidth
\setlength{\tikzwidth}{0.7\textwidth}
\definecolor{mycolor1}{rgb}{0,0.5,0}

\ifpdf
 \ifdraft
    \usepackage[draft]{pdfpages}%   externe PDF Seiten einbinden nur bei PDF-Latex möglich
    \else
    \usepackage[final]{pdfpages}%   externe PDF Seiten einbinden nur bei PDF-Latex möglich
 \fi                            %   Befehl: \includepdf{PDF-Datei} soll die Datei im Querformat angezeigt werden
    \else
\fi                             %   \includepdf[landscape=true]{PDF-Datei}

%---------------------------------------------------------------
% Debug-Ausgabe der Labels und References
%---------------------------------------------------------------
\ifdraft
  \usepackage{showkeys} % Label werden im Draft Modus im Text angezeigt
\fi

%---------------------------------------------------------------
% Abweichende PostScript-Schriftarten
% werden in der main Datei ausgewählt, hier nichts ändern
%---------------------------------------------------------------
\ifthenelse{\equal{ 1 }{ \value{schrift} }}
    {
    \usepackage{mathptmx}          %   Times als Hauptschriftart, keine mathematische Fettschrift
    \usepackage{amssymb}
    \usepackage{amsbsy}
    \usepackage{amsmath}
    \usepackage{amsfonts}
    \usepackage{amstext}
    \renewcommand{\boldsymbol}[1]{\mathbf{#1}} % nur bei mathptmx !
                                               % damit boldsymbol wenigstens fette aufrechte Buchstaben macht
    }{}
\ifthenelse{\equal{ 2 }{ \value{schrift} }}
    {
    \usepackage[lf, minionint]{MinionPro} %   OpenType von Adobe, mathematische Fettschrift vorhanden, Schrift ist
                                %   ist im Standard-LaTeX nicht installiert.
                                %   Ist ein wenig Aufwand das Ganze zu installieren, Matthias Dagen fragen
    }{}
\ifthenelse{\equal{ 3 }{ \value{schrift} }}
    {
    \usepackage{mathpazo}          %   nette Buchschrift aber sehr gross, mathematische Fettschrift vorhanden
    \usepackage{amssymb}
    \usepackage{amsbsy}
    \usepackage{amsmath}
    \usepackage{amsfonts}
    \usepackage{amstext}
    }{}
\ifthenelse{\equal{ 4 }{ \value{schrift} }}
    {
    \usepackage{times}
    \usepackage{amssymb}
    \usepackage{amsbsy}
    \usepackage{amsmath}
    \usepackage{amsfonts}
    \usepackage{amstext}
    }{}
\ifthenelse{\equal{ 5 }{ \value{schrift} }}
    {
    \usepackage{amssymb}
    \usepackage{amsbsy}
    \usepackage{amsmath}
    \usepackage{amsfonts}
    \usepackage{amstext}
    \usepackage{lmodern}
    }{}
\usepackage[scaled=.92]{helvet} %   11pt Helvetica für Überschriften etc. etwas kleiner da, Helvetica an sich größer ist
%\usepackage{courier}            %   Courier bei \texttt
%\usepackage{upgreek}            %   Aufrechte griechische Buchstaben

% ------------------------------------------------------------------------------------------
% Bild- und Tabellentitel FETT
% ------------------------------------------------------------------------------------------
%\def\figurename{\bfseries Bild}
%\def\tablename{\bfseries Tabelle}

%%Andere Beschreibung von figure
\addto\captionsngerman{
	\renewcommand{\figurename}{\bfseries Bild}%%Andere Beschreibung von figure
	\renewcommand{\listfigurename}{Bildverzeichnis}
	\renewcommand{\tablename}{\bfseries Tabelle}
}

%---------------------------------------------------------------
% Quelltexte formatieren
%---------------------------------------------------------------
\usepackage{listings}


%\lstloadlanguages{
		%C,
		%C++,
		%XML
%}

\lstset{
		language=XML,
		basicstyle=\footnotesize\ttfamily, % Standardschrift
		numbers=left,               % Ort der Zeilennummern
		numberstyle=\tiny,          % Stil der Zeilennummern
		numbersep=5pt,              % Abstand der Nummern zum Text
		tabsize=2,                  % Groesse von Tabs
		extendedchars=true,         %
		breaklines=true,            % Zeilen werden Umgebrochen        
		keywordstyle=\color{Red2},
		frame= b,         
		stringstyle=\color{Purple2}\ttfamily, % Farbe der String
		showspaces=false,           % Leerzeichen anzeigen ?
		showtabs=false,             % Tabs anzeigen ?
		xleftmargin=17.5pt,
		framexleftmargin=17pt,
		framexrightmargin=5pt,
		framexbottommargin=4pt,
		linewidth= \dimexpr\textwidth-2\fboxsep-2\fboxrule,
		comment=[l]{\#},
		morecomment=[s]{<!--}{-->},
		commentstyle=\color{Green4},
		%backgroundcolor=\color{grey},
		showstringspaces=false,      % Leerzeichen in Strings anzeigen ?        
		morekeywords={__global__, name, pkg, args, type, from, to, textfile, respawn, value, output, radius, ixx, ixy, ixz, iyy, iyz, xyz, rpy, reference},  % CUDA specific keywords
		aboveskip = 18pt, 
		belowskip = 18pt
}

\usepackage{caption}
\DeclareCaptionFont{white}{\color{white}}
\DeclareCaptionFormat{listing}{\colorbox{darkgray}{\parbox{\dimexpr\textwidth-2\fboxsep-2\fboxrule}{\hspace{15pt}#1#2#3}}}
\captionsetup[lstlisting]{format=listing,labelfont=white,textfont=white, singlelinecheck=false, margin=0pt, font={bf,footnotesize}}

%---------------------------------------------------------------
% ein paar Längen einstellen
%---------------------------------------------------------------
%\setlength{\parskip}{1ex plus0.5ex minus0.2ex} % Absatzabstand etwas größer
\setlength{\itemsep}{0ex plus0.2ex}            % Abstand zweier Listenelemente kleiner
\setlength{\parindent}{0ex}                     % kein Absatzeinzug
\setlength{\belowcaptionskip}{0.4cm}            % Abstand caption - Tabelle größer
\setlength{\abovecaptionskip}{0.4cm}            % Abstand caption - Tabelle größer

%---------------------------------------------------------------
% Kopf- und Fusszeilen
%---------------------------------------------------------------
\usepackage[plainheadsepline,  % linie zw. kopf und text auch bei seitenstil "plain"
           %headexclude,       % kopfzeile gehört nicht zum textkörper
           %footexclude        % fusszeile gehört nicht zum textkörper
          ]       
           {scrpage2}
\pagestyle{scrheadings}
\clearscrheadfoot              % voreinstellungen loeschen
\ihead{\normalfont\headmark}   % kolumnentitel innen
\ohead[\pagemark]{\pagemark}   % seitenzahl aussen

%---------------------------------------------------------------
% Nummerierungs-Tiefe des Inhaltsverzeichnis und der Abschnitte einstellen
%---------------------------------------------------------------
\setcounter{secnumdepth}{2} \setcounter{tocdepth}{2}

%---------------------------------------------------------------
% Abkürzungsverzeichnis
%---------------------------------------------------------------
% mit dem Befehl \nomenclature[l/g/a]{Abkürzung}{Bezeichnung} kann direkt im Code die Abkürzung eingefügt werden
% das Paket sortiert diese und fügt sie mit dem Befehl \printnomenclature ein
% l, g, oder a steht dabei für die Gruppierung Lateinische bzw. Griechische Buchstaben, Abkürzungen
% Nach Einfügen neuer Abkürzungen muss folgender Befehl in der Eingabeaufforderung im Tex-Verzeichnis eingegeben werden:
% -> vorher einmal kompilieren
% -> makeindex main.nlo -s nomencl.ist -o main.nls
% -> nochmal kompilieren
%\makeindex
\usepackage{nomencl}
% \let\abbrev\nomenclature
\renewcommand{\nomname}{Abkürzungsverzeichnis}
\setlength{\nomlabelwidth}{.25\hsize}
\renewcommand{\nomlabel}[1]{#1 \hfill}
\setlength{\nomitemsep}{-\parsep}

\renewcommand{\nomgroup}[1]{%
	\ifthenelse{\equal{#1}{L}}{\item[\textbf{Lateinische Buchstaben}]}
	{%
		\ifthenelse{\equal{#1}{G}}{\item[\textbf{Griechische Buchstaben}]}
		{%
			\ifthenelse{\equal{#1}{A}}{\item[\textbf{Abkürzungen}]}
			{
				\ifthenelse{\equal{#1}{K}}{\item[\textbf{Koordinatensysteme}]}{}
			}
		}
	}
}

\makenomenclature

%---------------------------------------------------------------
% Hyperlinks für pdfTeX
%---------------------------------------------------------------
\ifpdf
    % hier stehen befehle, die nur für pdftex gelten
    \usepackage[pdfpagelabels,  % logische (z.b. auch roemische) seitenzahlen
                bookmarks,       % Bookmarks für die einzelnen Abschnitte
                pdftex
                ]{hyperref}
    \hypersetup{
    %   colorlinks,  % Links mit farbigem Text
        pdfborder   = 0 0 0,
        plainpages  = false,
        bookmarksnumbered = true,
    %   bookmarksopen
    }
\else
    % hier stehen befehle, die nur für latex gelten
    \usepackage{hyperref} % hier ohne colorlinks und pdf-krams
\fi
%\usepackage{url}

%extra inhaltsverzeichnisse möglich
\usepackage[nohints]{minitoc}     
\renewcommand{\mtctitle}{Anhangsverzeichnis} 				% Name ändern            
\renewcommand{\mtifont}{\large\bfseries\sffamily}		% Titel font ändern
\renewcommand{\mtcSfont}{\rm}												% Section einträge normal
\mtcsetrules{minitoc}{off}													% Linien ausmachen

%---------------------------------------------------------------
% SVGs einfügen
%---------------------------------------------------------------
%% SVG to TeX
% from InkscapePDFLaTeX.pdf
% by Johan Engelen, 2010
% Information: 
% http://tug.ctan.org/tex-archive/info/svg-inkscape
% -shell-escape muss im Ausgabeprofil stehen
% inkscape.exe muss im Path-Folder sein

% Funktion zum ueberpruefen auf Aenderung
\newcommand{\executeiffilenewer}[3]{%
                \ifnum\pdfstrcmp{\pdffilemoddate{#1}}%
                {\pdffilemoddate{#2}}>0%
                {\immediate\write18{#3}}\fi%
}
% Wenn Aenderung dann TeX-Export ausfuehren und einbinden
%\newcommand{\includesvg}[1]{%
                %\executeiffilenewer{#1.svg}{#1.pdf}%
                %{inkscape -z -C --file=#1.svg %
                %--export-pdf=#1.pdf --export-latex}%
                %\input{#1.pdf_tex}%
%}

%% Include svg mit Text aus textext plugin
% text wird im svg mit textext hinzugefügt.
% falls Änderungen im .svg gefunden werden, wird das Bild 
% neu exportiert und eingefügt.
% \includesvg{ordner/file} /ohne endung
\newcommand{\includesvg}[4][]{
			\executeiffilenewer{#3.svg}{#3.pdf}
			{inkscape -z -D --file=#3.svg
			--export-pdf=#3.pdf}
			\begin{figure}[#2]
				\centering
				\includegraphics[#1]{#3}
				\caption{#4} \label{fig.#3}
				\vspace*{-3mm}
			\end{figure}
}

\newcommand{\includesinglesvg}[2][]{
			\executeiffilenewer{#2.svg}{#2.pdf}
			{inkscape -z -D --file=#2.svg
			--export-pdf=#2.pdf}
			\includegraphics[#1]{#2}
}

\usepackage[right]{eurosym}

%---------------------------------------------------------------
% Andreas Serov hinzugefügte Pakete
%---------------------------------------------------------------
\usepackage{bm}					% bold math symbols
\usepackage{tikz}				% graphics via tikz

\usepackage{lipsum} 			% zum Einfügen von Paragraphen von text
\usepackage{wrapfig}			% Bild und Textfluss, gibt wohl bessere Alternativen

\usepackage{subcaption}		% für subfigures, package "subfigure" deprecated

\usepackage{gensymb}

\usepackage{dsfont}				% für mathematische Buchstaben wie Erwartungswert E oder reelle Zahlen R
\usepackage{algorithm}			% for algorithm environment - caption, label, etc
%\usepackage{algorithmicx}
\usepackage{algpseudocode}		% for algorithmic environment
\floatname{algorithm}{Algorithmus} % Algorithm in Algorithmus umbenennen in algorithm environment

\usepackage{mathtools}	% \coloneeqq := 

%\usepackage{float} % zum exakten Platzieren von Bildern etc

%---------------------------------------------------------------
% Andreas Serov tikz befehle und Farben
%---------------------------------------------------------------
% custom colors
\definecolor{imesblue}{rgb}{0.00000,0.313725,0.607843}			
\definecolor{imesorange}{rgb}{0.90588235,0.48235,0.160784}
\definecolor{imesgreen}{rgb}{0.784313,0.82745,0.090196}%
%% tikz libraries und styles, die allgemein für Grafiken verwendet werden
\usetikzlibrary{shapes,arrows, positioning}
% kreis, rechteck und pfeil mit in imesblau
\tikzstyle{simplecircle} = [draw, circle, node distance=2.5cm, line width=3pt, color=imesblue]
\tikzstyle{simplebox} = [draw, rectangle, node distance=2.5cm, line width=3pt, color=imesblue, rounded corners]
\tikzstyle{simplearrow} = [draw, -latex, line width=2pt, color=imesblue]

%---------------------------------------------------------------
% Andreas Serov hinzugefügte Kommandos
%---------------------------------------------------------------
\newcommand{\shortminus}{\scalebox{0.75}[1.0]{\,\( - \)\,}} % kurzes minuszeichen
\newcommand{\shortequal}{\,{=}\,}	% = zeichen mit weniger abstant

\DeclareMathOperator*{\argmin}{argmin} % \argmin operator
\DeclareMathOperator*{\argmax}{argmax} % \argmax operator

\renewcommand{\algorithmicrequire}{\textbf{Eingang:}} % Eingang: anstatt englisches Require
